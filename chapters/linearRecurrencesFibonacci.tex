\section{Linear recurrences - The Fibonacci sequence}\label{fibonacci} 

Hallo

\subsection{Fibonacci's rabbit problem}

\subsection{Fibonacci numbers and the Pascal triangle}

\subsection{Domino tilings}

\subsection{Linear recurrence relations}

\subsection{The characteristic equation}

\subsection{Linear recurrence relations of order $2$}

\subsection{The Binet formula}

\subsection{Linear recurrence relations of arbitrary order}

\subsection{The case of roots with multiplicities}

\subsection{Exercises}

\begin{enumerate}
\item Which of the following is a Fibonacci sequence?
\begin{enumerate}
\item The number of partitions, i.e. the presentations of a natural number as a sum of positive non-increasing summands (i.e., $3=2+1=1+1+1$)
\item The number of sequences of $0$'s and $1$'1 with $n$ digits that contain no two consecutive zeroes
\item The number of subsets of $\{1,2, \cdots, n\}$ that contain no consecutive integers
\item The number of partitions of a rectangle $2 \times n$ into rectangles $2 \times 1$
\item The number of compositions of a natural number into positive odd summands (i.e., $4=1+1+1+1=1+3=3+1$)
\end{enumerate}
\item The Fibonacci sequence can be continues "backwards" using the same rule: $F_n=F_{n-1}+F_{n-2}$. For example, $F_0=0, F_{-1}=1$. Find $F_{-10}$.
\item Find the maximal common ration of a geometric progression $a_n$ satisfying the following equation: $a_{n+2}=3a_{n+1}-2a_n$.
\item The sequence $a_n$ is defined by the recurrence relation $a_{n+3}=3a_{n+2}-3a_{n+1}+a_n$ with initial values $a_0=a_1=0, a_2=1$. Find $a_{100}$.
\item The sequence $a_n$ is defined by the recurrence relation $a_{n+4}=a_{n+3}-a_{n+2}+a_{n+1}-a_n$ with initial values $a_0=1607, a_1=1707, a_2=1814, a_3=1914$. Find $a_{100}$.
\end{enumerate}