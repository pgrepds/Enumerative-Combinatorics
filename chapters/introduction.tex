\section{Introduction}\label{introduction} 

We start by discussing fundamental combinatorial constructions such as words of a fixed alphabet, permutations of finite
sets and the number of subsets with a given number of elements. While we give some examples of how some combinatorial problems 
naturally arise, our main goal in this first section is to provide an intuitive introduction to the topic of 
Enumerative Combinatorics. 

\subsection{Words}

The first definition of this lecture is the definition of a word over a finite alphabet of symbols.

\begin{defn}[word]
Let $A$ be a finite set, called \textit{alphabet}. A \textit{word} is a finite sequence of elements of $A$. The \textit{size} of $A$ is its cardinality $|A|$.
The \textit{size} or \textit{length} of a word is its number of elements from $A$.
\end{defn}
\noindent
We might abuse language to a certain extent and call the elements of $A$ \textit{letters}.

\begin{exmp}
Let $A=\{1, 2, 3, 4\}$, then $112$, $132$, $1234$, $111$, $2345$, and so on, are all words over $A$.
Notice that the length of the words is not fixed.    
\end{exmp}

In the study of combinatorics a very classical question is as follows. Given an alphabet $A$ of length $|A|=n$ , how many words over $A$ of length $k$ exist? This question leads us to the following theorem.

\begin{theorem}
The number of words of length $k$ in an alphabet consisting of $n$ letters is equal to $n^k$.
\label{thmNumWords}
\end{theorem}
\begin{proof}
There are $n$ possibilities for the first letter. The same holds for all other letters.
Thus, the total number of words equals
$$
\underbrace{n \cdot n \cdot n \cdots n}_\text{$k$-times} = n^k
$$
\end{proof}

The above proof is fairly simple, but there is a different possible interpretation of this problem. Let $X$ and $Y$ be sets. We might ask: what is the total
number of maps of the form $X \to Y$ without restrictions (an element in $Y$ can be hit multiple times)? It appears that this question can be solved using 
Theorem \ref{thmNumWords}. For seeing this, we number the elements of $X$ and $Y$ using natural numbers.
Given a map $f:X \to Y$, the sequence $f(1),f(2), \cdots, f(k)$ forms a word of length $k$. The number of words is therefore equivalent to the number of such maps.

\subsection{Permutations}

A notion that is closely related to the definition of a word, is the definition of a \textit{permutation}. First of all, the basic idea is to count maps, but 
usually with certain properties. Let $X$ and $Y$ be sets with $|X|=k$ and $|Y|=n$. We might ask: how many bijections $f:X \to Y$ are there? Recall that
a map is bijective if it is surjective and injective. In other words, given two elements $x,y \in X$ such that $f(x)=f(y)$, then it is implied that $x=y$ (the injective property), and 
every element of $Y$ appears in the preimage of a certain element of $X$ (the surjective property). It follows that the cardinalities of $X$ and $Y$ must be equal.
These observations lead us to the definition of a permutation.

\begin{defn}[permutation]
A \textit{permutation} is a bijection from $\{1, 2, 3, \cdots, n\}$ to itself.
\end{defn}
\noindent
In that sense the number of permutations is the number of bijections from $\{1, 2, \cdots, n\}$ to itself.
Equivalently, a permutation might be interpreted as a total linear ordering of the set $\{1, 2, 3, \cdots, n\}$.
If so, we need to arrange them in a certain way. How many ways to do this are there? 
The following theorem provides one possible solution.

\begin{theorem}
The number of permutations of $\{1, 2, \cdots, n\}$ is equal to $n!=1\cdot 2 \cdot 3 \cdots n$.\footnote{Recall that $0!=1$.}
\end{theorem}

\begin{proof}
There are $n$ choices for the first element. For the second element there are $n-1$ choices since
one element is already occupied. For the $k$-th element there are $n-k+1$ choices. For the last element there is one choice.
Thus, the total number of permutations is given by $n(n-1)(n-2)\cdots 2 \cdot 1=n!$. This completes the proof.
\end{proof}

In the following, we will generalize the above result to injective maps. Let $k \leq n$. In how many ways can we produce a collection 
of $k$ elements out of $n$? Such a collection is called a $k$-permutation.

\begin{defn}[$k$-permutation]
A $k$-permutation is a total linear ordering on a $k$-element subset of $\{1, 2, \cdots, n\}$.
\end{defn}

\begin{exmp}
Let $\{1,2 ,3\}$. A $1$-permutation is just a $1$-element subset, thus there is nothing order. The set of 
$2$-permutations is given by $\{\{1,2\}, \{1,3\}, \{2,1\}, \{2,3\}, \{3,1\}, \{3,2\}\}$.
Thus , there are six $2$-permutations in total.
\end{exmp}

The general case of the number of $k$-permutations is examined in the following theorem.

\begin{theorem}
The number of $k$-permutations of $\{1,2, \cdots, n\}$ is equal to $$n(n-1)\cdots (n-k+1).$$
\end{theorem}

\begin{proof}
There are $n$ possible choices for the first entry of the subset. There are $(n-1)$ possible choices for the second element because the second element can occupy any entry except the one of the first element. 
For the $k$-th entry there are $n-k+1$ possible choices. All these possibilities are independent, hence
the total number of $k$-permutations is equal to their product $n(n-1)\cdots (n-k+1)$.
\end{proof}

The number $n(n-1)\cdots (n-k+1)$ is sometimes called the $k$ \textit{decreasing power} and is denoted by $n^{\downarrow k}$.

\subsection{Merry-go-rounds and Fermat's little theorem}
The problem which we are going to discuss next will illustrate a connection of combinatorics to algebra and possibly number theory.
\\
Consider a train $Q$ consisting of $p$ carriages. Suppose that we have $n$ colours. In how many ways can we paint $Q$, if we can paint 
each carriage with one of the $n$ colours? Consider figure \ref{fig:marrygoroundtrain} as an example. There is a train with $5$ carriages which
are coloured in a certain way using $3$ different colours.

\begin{figure}[ht]
    \centering
    \incfig{marrygoroundtrain}
    \caption{Initial train colouring.}
    \label{fig:marrygoroundtrain}
\end{figure}

Now, this is exactly the word problem, thus the solution is given by $n^p$.
\\
\noindent
\textbf{Question:} Suppose that instead of a train we have a merry-go-round and suppose further, for some reason, that $p$ is prime. 
In how many ways can we paint each carriage of a merry-go-round using $n$ colours?

\begin{figure}[ht!]
    \centering
    \scalebox{.3}{\incfig{marrygoroundinitialcase}}
    \caption{Marry-go-round initial case.}
    \label{fig:marrygoroundinitialcase}
\end{figure}

In figure \ref{fig:marrygoroundinitialcase} a marry-go-round with $5$ carriages is visualized. They are again coloured using $3$ colours.\\
What is the difference to the problem we have considered before? In this case we can rotate the carriages without changing its colours, but these colourings are considered to be identical.
\\
\\
\textbf{Idea:} We cut the merry-go-round (consider the two lines in figure \ref{fig:marrygoroundinitialcase} as the "cutting point") and make a train out of it and there are exactly $p$ places on which we could make a cut.
\\
\\
The first naive guess would be that the solution to our problem is equal to $\frac{n^p}{p}$, since we can get $p$ trains from each merry-go-round. This is wrong for an obvious reason. The number $\frac{n^p}{p}$
is by no means always integer. So let us trying to understand what is wrong with this reasoning. First, if we cut at two different places in our merry-go-round, then we could get the same train. An easy way for seeing this
is to consider a merry-go-round in which each carriage is coloured using the same colour. There are still $p$ ways to make a train out of it but each of them is identical. Compare this reasoning with figure \ref{fig:marrygoroundonecolour}.

\begin{figure}[ht]
    \centering
    \incfig{marrygoroundonecolour2}
    \caption{Marry-go-round with only one colour.}
    \label{fig:marrygoroundonecolour}
\end{figure}

To say it in different words, if we rotate a merry-go-round by an arbitrary number of times, it will eventually coincide with itself, thus if a merry-go-round goes to itself after some nontrivial rotation, then all carriages have the same colour.
Now, notice that this is not the case if not all carriages have the same colour, since then we get exactly $p$ different trains, but observe that this is only true provided that $p$ is prime. In order to see this, consider a merry-go-round with $4$ carriages,
and assume that we have two colours. black and white. Then, there are only two different trains which we can get from this merry-go-round. Colour the merry-go-round as follows: black - white - black - white. Then after two rotations, we get the same train. Thus, 
instead of four trains we get two. However, if $p$ is prime, then our claim is indeed true.
\newpage
If our merry-go-round has carriages of different colours, then we can compute the number of such merry-go-rounds.
It is the number of trains divided by $p$. What is the number of trains which have different colors?
This is $n$ to the power of $p$, which is the total number of trains, minus the number of trains of the same color, which is just $n$ and for each color, there is only one train for which all carriages have this given colour.
Hence, our final result is given by the following equation.
$$
\frac{n^p-n}{p} + p
$$

It is again surprising that the above number is always an integer. This is not obvious at all and proving this using number theory is quite more involved. In general, we have obtained a very nice result given by the following theorem.

\begin{theorem}
Given $n \in \mathbb{N}$ and a prime $p$, then $n^p -n\ |\ p$ ($n^p - n$ is divisible by $p$).  
\end{theorem}
\noindent
The above is Fermat's little Theorem and we have proven it in a completely combinatorial way.

\subsection{Binomial coefficients}

Next, we consider the following important definition.

\begin{defn}[bionmial coefficient]
Let $k \leq n$. The number of $k$-element subsets of $\{1, 2, \cdots, n\}$ is called a \textit{binomial coefficient} ${n \choose k}$ ("$n$ choose $k$").
\end{defn}

How do we interpret this number? Suppose that we are hiring people to certain positions and suppose that we have $n$ candidates and we have $k$ identical positions.
The number of selecting $k$ people out of $n$ is called ${n \choose k}$. Notice that this is different than the problem from problem about $k$-permutations.
In our current case, the order of the $k$ elements that we choose does not matter. Thus, choosing all $k$ element subsets of a set and then ordering them in all possible ways
is equal to the $k$-permutations that we have seen above. How do we count the number of $k$-element subsets of a set?

\begin{theorem}
The number of $k$-element subsets of a set with cardinality $n$ is equal to 
$$
{n \choose k}=\frac{n!}{k!(n-k)!}=\frac{n(n-1)\cdots (n-k+1)}{k!}.
$$
\label{thmBinomial}
\end{theorem}

\begin{proof}
We consider ordered collections of distinct elements of length $k$. There are $n$ ways of choosing the first element, and $n-k+1$ ways of choosing the $k$-th element.
Hence there are 
$$
n(n-1)\cdots (n-k+1)=\frac{n!}{(n-k)!}
$$
ways such ordered collections. Since we are concerned with the number of unordered subsets and each subset has cardinality $k$, each subset can be rearranged in $k!$ ways.
Therefore, the number of ways to choose a $k$-element subset of a set with cardinality $n$ is given by 
$$
\frac{\frac{n!}{(n-k)!}}{k!}=\frac{n!}{k!(n-k)!}.
$$
This completes the proof.
\end{proof}

With the above theorem, we have shown something additionally that is not obvious at all and not necessarily easy to prove either.
We have also shown that the number $\frac{n(n-1)\cdots (n-k+1)}{k!}$ is integer.

At this point we provide some properties of binomial coefficients. We leave the proof as an easy exercise to the reader.

\begin{enumerate}[(i)]
    \item $\displaystyle {n \choose 0}={n \choose n}=1$.
    \item $\displaystyle {n \choose 1}={n \choose n-1}=n$
    \item $\displaystyle {n \choose 1}={n \choose n-k}$.
    \item $\displaystyle {n \choose k}={n - 1 \choose k -1} + {n - 1 \choose k}$.
\end{enumerate}

Let us examine property $(iv)$ in more detail. Despite its obvious algebraic proof, there is an intuitive combinatorial version as well.
We provide a proof sketch.

\begin{theorem}
$\displaystyle {n \choose k}={n - 1 \choose k -1} + {n - 1 \choose k}$
\label{recurrencePascal}
\end{theorem}

\begin{proof}[Proof-sketch]
Given a set $A=\{1, \cdots, n\}$ with $n$ elements. The number of $k$-element subsets of $A$ is ${n \choose k}$ according to theorem \ref{thmBinomial}.
Suppose that we pick the element $1$ of $A$, then all $k$-element subsets of $A$ are divided into two parts, some of them contain the element $1$, and the others do not.
Since we have already picked one element, we need to pick $k-1$ elements out of the remaining $n-1$ elements. There are ${n-1 \choose k-1}$ subsets containing $1$.
Thus, the subsets of cardinality $k$ that do not contain $1$ is equal to ${n - 1 \choose k}$, because we want to pick a $k$-element subset out of $n-1$ elements since $1$ element is forbidden.
\\
Adding up the two parts are equal to ${n \choose k}$. This completes the proof.
\end{proof}


\subsection{The Pascal triangle}

Binomial coefficients can be arranged in a triangle which is called the Pascal triangle.
Its rows are indexed by integers from $0$ up to infinity and in each row we write down all binomial coefficients.
For example, for ${n \choose k}$, $n=4$ represents the fifth row of the Pascal triangle and since $k \leq n$ there are $5$ binomial coefficients, namely ${4 \choose 0}=1, {4 \choose 1}=4, {4 \choose 2}=6, {4 \choose 3}=4$ and ${4 \choose 4}=1$.
\\
The relation given In Theorem \ref{recurrencePascal} provides a way of constructing the Pascal triangle. It says that the sum of two neighbors, namely ${n-1 \choose k-1}$ and ${n-1 \choose k}$ is equal to ${n \choose k}$. Thus, taking two neighbors in one row and
adding them up gives as the element below them (e.g. $3+3$ in row $n=3$ provides the element $6$ in the row $n=4$).
\\
\\
\begin{tabular}{>{$n=}l<{$\hspace{12pt}}*{13}{c}}
    0 &&&&&&&1&&&&&&\\
    1 &&&&&&1&&1&&&&&\\
    2 &&&&&1&&2&&1&&&&\\
    3 &&&&1&&3&&3&&1&&&\\
    4 &&&1&&4&&6&&4&&1&&\\
    5 &&1&&5&&10&&10&&5&&1&\\
    6 &1&&6&&15&&20&&15&&6&&1
\end{tabular}

\begin{proposition}
$\displaystyle {n \choose 0} + {n \choose k} + \cdots + {n \choose n}=2^n.$
\label{propositionPascalTriangle}
\end{proposition}

\begin{proof}
The binomial coefficient ${n \choose k}$ is the number of $k$-element subsets of a $n$-element set $A=\{1, \cdots, n\}$. Hence,
$$
\sum_{k=0}^n {n \choose k}
$$
is equal to the number of all possible subsets of $A$. We can associate a subset of $A$ with a binary number of length $n$ 
whereby each digit in this binary number represents a specific element in $A$. Now, setting a digit to $1$ means that we pick the element from $A$ at this position
for our subset. Setting it to $0$ means that we do not pick it. This is equal to the number of maps $\{1, \cdots, n\} \to \{0,1\}$ and the total number of such maps is equal to $2^n$ or in other words, the number of binary numbers that
can be represented with $n$ digits is equal to $2^n$. This completes the proof.
\end{proof}

We call the above binomial coefficients, because they arise as coefficients in the binomial formulas $(a+b)^n$. Let us prove this.

\begin{theorem}(Newton's Binomial Theorem)
$$
(a+b)^n = \sum_{k=0}^n {n \choose k} a^{n-k} \cdot b^k=a^n+n \cdot a^{n-1} \cdot b+ \frac{n(n-1)}{2} a^{n-2} \cdot b^2+ \cdots + n \cdot a \cdot b^{n-1}+b^n
$$
\label{newtonBinomialThm}
\end{theorem}

\begin{proof}
We expand the left-hand-side.
$$
(a+b)^n=\underbrace{(a+b)(a+b)(a+b)\cdots (a+b)}_{n-times}
$$
Every term in the above sum is the result of choosing either $a$ or $b$ from each of the $n$ factors. We want to calculate the coefficient for $a^{n-k}b^k$.
Since $b^k$, we must choose $b$ from exactly $k$ of the factors and we choose $a$ from the remaining $n-k$ factors.
Thus, the coefficients for $a^{n-k}b^k$ is equal to ${n \choose k}$. This completes the proof.
\end{proof}

\begin{corolarry}
$\displaystyle {n \choose 0}- {n \choose 1} + {n \choose 2} - {n \choose 3} + \cdots + (-1)^n {n \choose n}=0$
\end{corolarry}

\begin{proof}
Consider the binomial formula as provided in Theorem \ref{newtonBinomialThm}. Now, let $a=1$ and $b$ (or vice versa), it follows that 
$$
(1-1)^n= \sum (-1)^k {n \choose k}.
$$
\end{proof}

Notice, that Newton's binomial Theorem provides an easy proof for proposition \ref{propositionPascalTriangle} as well by considering $(1+1)^n=\sum {n \choose k}$.

\subsection{Exercises}

\begin{enumerate}
    \item Find the number of subsets of a $10$-element set with an odd number of elements.
    \item Compute the sum $\displaystyle \sum_{k=0}^n 2^k {n \choose k}$.
    \item Find the number of ways to rearrange the letters of the word \textit{EULER}.
    \item Find the number of ways to rearrange the letters of the word \textit{CATALAN}.
    \item How many necklaces can be formed with $7$ distinct beads? Necklaces can be rotated and/or flipped over. All beads must be used.
\end{enumerate}