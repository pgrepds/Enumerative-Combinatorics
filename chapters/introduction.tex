\section{Introduction}\label{introduction} 

We start by discussing fundamental combinatorial constructions such as words of a fixed alphabet, permutations of finite
sets and the number of subsets with a given number of elements. While we give some examples of how some combinatorial problems 
naturally arise, our main goal in this first section is to provide an intuitive introduction to the topic of 
Enumerative Combinatorics. 

\subsection{Words}

The first definition of this lecture is the definition of a word over a finite alphabet of symbols.

\begin{defn}[word]
Let $A$ be a finite set, called \textit{alphabet}. A \textit{word} is a finite sequence of elements of $A$. The \textit{size} of $A$ is its cardinality $|A|$.
The \textit{size} or \textit{length} of a word is its number of elements from $A$.
\end{defn}
\noindent
We might abuse language to a certain extent and call the elements of $A$ \textit{letters}.

\begin{exmp}
Let $A=\{1, 2, 3, 4\}$, then $112$, $132$, $1234$, $111$, $2345$, and so on are all words over $A$.
Notice that the length of the words is not fixed.    
\end{exmp}

In the study of combinatorics a very classical question is as follows. Given an alphabet $A$ of length $|A|=n$ , how many words of length $k$ consisting of $n$
letters exist? This question leads us to the following theorem.

\begin{theorem}
The number of words of length $k$ in an alphabet consisting of $n$ letters is equal to $n^k$.
\label{thmNumWords}
\end{theorem}
\begin{proof}
There are $n$ possibilities for the first letter. The same holds for all other letters.
Thus, the total number of words equals
$$
\underbrace{n \cdot n \cdot n \cdots n}_\text{$k$-times} = n^k
$$
\end{proof}

The above proof is fairly simple, but there is a different possible interpretation of this problem. Let $X$ and $Y$ be sets. We might ask: what is the total
number of maps of the form $X \to Y$ without restrictions (an element in $Y$ can be hit multiple times)? It appears that this question can be solved using 
theorem \ref{thmNumWords}. For seeing this, we number the elements of $X$ and $Y$ using natural numbers.
Given a map $f:X \to Y$, the sequence $f(1),f(2), \cdots, f(k)$ forms a word of length $k$. The number of words is therefore equivalent to the number of such maps.


\subsection{Permutations}

\subsection{Merry-go-rounds and Fermat's little theorem}

\subsection{Binomial coefficients}

\subsection{The Pascal triangle}

\subsection{Exercises}