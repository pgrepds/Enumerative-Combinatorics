\section{Introduction}\label{introduction} 

We start by discussing fundamental combinatorial constructions such as words of a fixed alphabet, permutations of finite
sets and the number of subsets with a given number of elements. While we give some examples of how some combinatorial problems 
naturally arise, our main goal in this first section is to provide an intuitive introduction to the topic of 
Enumerative Combinatorics. 

\subsection{Words}

The first definition of this lecture is the definition of a word over a finite alphabet of symbols.

\begin{defn}[word]
Let $A$ be a finite set, called \textit{alphabet}. A \textit{word} is a finite sequence of elements of $A$. The \textit{size} of $A$ is its cardinality $|A|$.
The \textit{size} or \textit{length} of a word is its number of elements from $A$.
\end{defn}
\noindent
We might abuse language to a certain extent and call the elements of $A$ \textit{letters}.

\begin{exmp}
Let $A=\{1, 2, 3, 4\}$, then $112$, $132$, $1234$, $111$, $2345$, and so on are all words over $A$.
Notice that the length of the words is not fixed.    
\end{exmp}

In the study of combinatorics a very classical question is as follows. Given an alphabet $A$ of length $|A|=n$ , how many words of length $k$ consisting of $n$
letters exist? This question leads us to the following theorem.

\begin{theorem}
The number of words of length $k$ in an alphabet consisting of $n$ letters is equal to $n^k$.
\label{thmNumWords}
\end{theorem}
\begin{proof}
There are $n$ possibilities for the first letter. The same holds for all other letters.
Thus, the total number of words equals
$$
\underbrace{n \cdot n \cdot n \cdots n}_\text{$k$-times} = n^k
$$
\end{proof}

The above proof is fairly simple, but there is a different possible interpretation of this problem. Let $X$ and $Y$ be sets. We might ask: what is the total
number of maps of the form $X \to Y$ without restrictions (an element in $Y$ can be hit multiple times)? It appears that this question can be solved using 
theorem \ref{thmNumWords}. For seeing this, we number the elements of $X$ and $Y$ using natural numbers.
Given a map $f:X \to Y$, the sequence $f(1),f(2), \cdots, f(k)$ forms a word of length $k$. The number of words is therefore equivalent to the number of such maps.

\subsection{Permutations}

A notion that is closely related to the definition of a word, is the definition of a \textit{permutation}. First of all, the basic idea is to count maps, but 
usually with certain properties. Let $X$ and $Y$ be sets with $|X|=k$ and $|Y|=n$. We might ask: how many bijections $f:X \to Y$ are there? Recall that
a map is bijective if it is surjective and injective. In other words, given two elements $x,y \in X$ such that $f(x)=f(y)$, then it is implied that $x=y$ (the injective property), and 
every element of $Y$ appears in the preimage of a certain element of $X$ (the surjective property). It follows that the cardinalities of $X$ and $Y$ must be equal.
These observations lead us to the definition of a permutation.

\begin{defn}[permutation]
A \textit{permutation} is a bijection from $\{1, 2, 3, \cdots, n\}$ to itself.
\end{defn}
\noindent
In that sense the number of permutations is the number of bijections from $\{1, 2, \cdots, n\}$ to itself.
Equivalently, a permutation might be interpreted as a total linear ordering of the set $\{1, 2, 3, \cdots, n\}$.
If so, we need to arrange them in a certain way. How many ways to do this are there? 
The following theorem provides one possible solution.

\begin{theorem}
The number of permutations of $\{1, 2, \cdots, n\}$ is equal to $n!=1\cdot 2 \cdot 3 \cdots n$.\footnote{Recall that $0!=1$.}
\end{theorem}

\begin{proof}
There are $n$ choices for the first element. For the second element there are $n-1$ choices since
one element is already occupied. For the $k$-th element there are $n-k+1$ choices. For the last element there is one choice.
Thus, the total number of permutations is given by $n(n-1)(n-2)\cdots 2 \cdot 1=n!$. This completes the proof.
\end{proof}

In the following, we will generalize the above result to injective maps. Let $k \leq n$. In how many ways can we produce a collection 
of $k$ elements out of $n$? Such a collection is called a $k$-permutation.

\begin{defn}[$k$-permutation]
A $k$-permutation is a total linear ordering on a $k$-element subset of $\{1, 2, \cdots, n\}$.
\end{defn}

\begin{exmp}
Let $\{1,2 ,3\}$. A $1$-permutation is just a $1$-element subset, thus there is nothing order. The set of 
$2$-permutations is given by $\{\{1,2\}, \{1,3\}, \{2,1\}, \{2,3\}, \{3,1\}, \{3,2\}\}$.
Thus , there are six $2$-permutations in total.
\end{exmp}

\newpage

The general case of the number of $k$-permutations is examined in the following theorem.

\begin{theorem}
The number of $k$-permutations of $\{1,2, \cdots, n\}$ is equal to $$n(n-1)\cdots (n-k+1).$$
\end{theorem}

\begin{proof}
There are $n$ possible choices for the first entry of the subset. There are $(n-1)$ possible choices for the second element because the second element can occupy any entry except the one of the first element. 
For the $k$-th entry there are $n-k+1$ possible choices. All these possibilities are independent, hence
the total number of $k$-permutations is equal to their product $n(n-1)\cdots (n-k+1)$.
\end{proof}

The number $n(n-1)\cdots (n-k+1)$ is sometimes called the $k$ \textit{decreasing power} and is denoted by $n^{\downarrow k}$.

\subsection{Merry-go-rounds and Fermat's little theorem}

The problem which we are going to discuss next will illustrate a connection of combinatorics to algebra and number theory.

\subsection{Binomial coefficients}

Next, we consider the following important definition.

\begin{defn}[bionmial coefficient]
Let $k \leq n$. The number of $k$-element subsets of $\{1, 2, \cdots, n\}$ is called a \textit{binomial coefficient} ${n \choose k}$ ("$n$ choose $k$").
\end{defn}

How do we interpret this number? Suppose that we are hiring people to certain positions and suppose that we have $n$ candidates and we have $k$ identical positions.
The number of selecting $k$ people out of $n$ is called ${n \choose k}$. Notice that this is different than the problem from problem about $k$-permutations.
In our current case, the order of the $k$ elements that we choose does not matter. Thus, choosing all $k$ element subsets of a set and then ordering them in all possible ways
is equal to the $k$-permutations that we have seen above. How do we count the number of $k$-element subsets of a set?

\begin{theorem}
The number of $k$-element subsets of a set with cardinality $n$ is equal to 
$$
{n \choose k}=\frac{n!}{k!(n-k)!}=\frac{n(n-1)\cdots (n-k+1)}{k!}.
$$
\end{theorem}

\begin{proof}
We consider ordered collections of distinct elements of length $k$. There are $n$ ways of choosing the first element, and $n-k+1$ ways of choosing the $k$-th element.
Hence there are 
$$
n(n-1)\cdots (n-k+1)=\frac{n!}{(n-k)!}
$$
ways such ordered collections. Since we are concerned with the number of unordered subsets and each subset has cardinality $k$, each subset can be rearranged in $k!$ ways.
Therefore, the number of ways to choose a $k$-element subset of a set with cardinality $n$ is given by 
$$
\frac{\frac{n!}{(n-k)!}}{k!}=\frac{n!}{k!(n-k)!}.
$$
This completes the proof.
\end{proof}

With the above theorem, we have shown something additionally that is not obvious at all and not necessarily easy to prove either.
We have also shown that the number $\frac{n(n-1)\cdots (n-k+1)}{k!}$ is integer.

\subsection{The Pascal triangle}

\subsection{Exercises}