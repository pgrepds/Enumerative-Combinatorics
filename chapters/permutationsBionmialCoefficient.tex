\section[Permutations, binomial coefficients]{Permutations and binomial coefficients continued}\label{permutationsBinomialCoefficient} 

The following lecture discusses some more problems in which binomial coefficients arise and will lead us 
to the famous principle of exclusion and inclusion.

\subsection{The cab driver problem}

We will begin with the so called 'cab driver problem'. Suppose that we have a city with streets arranged in 
a regular grid (like Manhattan) and suppose that our city contains $n$ blocks from north to south and $m$ blocks
from west to east. Now, suppose that we have a cab driver that wants to drive from one point on the grid to another
point, say from $A$ to $B$. In figure \ref{fig:cabdriverprobleminitialexample} the general setup is visualized.

\begin{figure}[ht]
    \centering
    \scalebox{.6}{\incfig{cabdriverprobleminitialexample}}
    \caption{The Cap Driver Problem.}
    \label{fig:cabdriverprobleminitialexample}
\end{figure}
\noindent
\textbf{Question:} In how many ways can we get from $A$ to $B$? We are only allowed to go east or north and we are not allowed to go backwards.
\\
\\
Let $P_{m,n}$ be the number of paths from $A$ to $B$. We notice first, that we can solve this problem by considering a sequence of letters. Let 
us denote "going north" by the letter "N" and "going east" by the letter "E", then a path from $A$ to $B$ corresponds to a sequence of N's and E's
of length $m+n$ with $n$ N's and $m$ E's. The total number of such words is equal to the total number of paths from $A$ to $B$.
\\
\\
There is also another solution. Starting from any point in the above grid, there are only two choices: go north or go east. Let $A$ be the starting point
and denote "going north in the first step" by $A'$ and "going east in the second step" by $A''$, then the total number of paths from $A$ to $B$ is equal to 
the total number of paths from $A'$ to $B$ plus the total number of paths from $A''$ to $B$. The result is a so called recurrence relation given by the following equation.
$$
P_{A \to B}=P_{A' \to B} + P_{A'' \to B}
$$
In general, if we go one step north the first time, then there are $P_{m, n-1}$ paths left from this position to $B$. Similarly, if we go east first, then there are $P_{m-1, n}$
paths left from this position to $B$. This results in the following relation.
$$
P_{m, n}=P_{m,n-1}+P_{m-1, n}
$$
Given the initial conditions $P_{0,n}=P_{m,0}=1$, and $P_{m,1}=m+1$ and $P_{1,n}=n+1$, our final solution can be found quickly.
$$
P_{m,n}={m + n \choose m}={m+n \choose n}
$$

\subsection{Balls in boxes and multisets}

Suppose that we have $k$ balls and $n$ boxes. In how many ways can we put $k$ balls into $n$ boxes? If the maximal capacity of each box is $1$, the answer would be 
$n \choose k$ if $n \geq k$ and $0$ otherwise. What happens if there are no restrictions on the capacity of boxes?
\\
This question leads us to the definition of a multiset. Let $X$ be a set.

\begin{defn}
A multiset is a function $\mu: X \to \mathbb{Z}_{\geq  0}$. The size of a multiset is $\sum_{x \in X} \mu (x)$.
\end{defn}

\begin{exmp}
Assume that we have $5$ boxes, that is $X=\{1, 2, 3, 4 ,5\}$, and $\mu(1)=1, \mu(2)=0, \mu(3)=3$ and $\mu(4)=1$. Compare this with figure \ref{fig:multisetboxesexample}. Notice that box $3$ contains 
$3$ balls. The size of $\mu$ is $5$.

\begin{figure}[ht]
    \centering
    \scalebox{0.4}{\incfig{multisetboxesexamples3}}
    \caption{Balls in boxes and mutlisets.}
    \label{fig:multisetboxesexample}
\end{figure}
\end{exmp}

It is easy to see that the number of configurations of $k$ balls into $n$ boxes corresponds to the number of multisets of $\{1,2,\cdots, n\}$
of size $k$. Now, let us interpret our problem in a slightly different way. Imagine $k$ balls and separators (visualized as "$|$") that correspond to a box. Then, the above configuration configuration 
looks as follows.
$$
0\ |\ \ |\ 000\ |\ 0
$$
For $4$ boxes, we only need $3$ separators, or in general for $n$ boxes we need $n-1$ separators. Thus, configurations of balls in boxes correspond to configurations of bars and balls.
If we consider balls and bars to be objects, then we have $k+n-1$ objects in total.
\begin{tikzcd}
    & k \text{ balls}  \\
k+n-1 \arrow[ru] \arrow[rd] &                  \\
    & n-1 \text{ bars}
\end{tikzcd}

The above investigations encourage the following theorem.

\begin{theorem}
$k$ balls can be put into $n$ boxes in ${n+k-1 \choose k}$ ways. Equivalently, the number of $k$-element multisets in $\{1, \cdots, n\}$ is ${n+k-1 \choose k}$.
\end{theorem}
\noindent
The formal proof is left as an exercise for the reader.

\subsection{Integer compositions}

We are going to discuss integer compositions in the following section. We start with the formal definition.

\begin{defn}
An integer composition of $n \in \mathbb{Z}_{> 0}$ is a presentation of $n$ as  an ordered sum $n=n_1 + n_2 + \cdots + n_k, n_1,\cdots,n_k \in \mathbb{Z}_{> 0}$.
\end{defn}

\begin{exmp}
For $n=1$ there is only one trivial presentation, that is just $1$ itself. For $n=2$ there is one presentation, namely $2=1+1$. For $n=3$ there are $4$ presentations,
 namely $3=2+1=1+2=1+1+1$. Notice that this sums are ordered, thus $1+2$ and $2+1$ are considered to be different.
 For $n=4$ there are $8$ presentations, namely $4=3+1=1+3=2+2=2+1+1=1+2+1=1+1+2=1+1+1+1$.
\end{exmp}

\begin{theorem} The following statements are true.
\begin{enumerate}[(i)]
    \item The number of compositions of $n$ is $2^{n-1}$.
    \item The number of compositions of $n$ with exactly $k$ summands is ${n-1 \choose k-1}$.
\end{enumerate}
\end{theorem}

\begin{proof}
We start with $(ii)$. The composition of $n$ with exactly $k$ summands corresponds to configurations of $n$ balls 
inside $k$ boxes, s.t. each box is nonempty. This is the same as the number of configurations of $n-k$ balls in $k$ boxes, 
which is equal to
$$
{n-k+k-1 \choose k-1} = {n-1 \choose k-1}.
$$
We continue with $(i)$. The answer is trivial. The total number of compositions follows from the following equation.
$$
{n-1 \choose 0} + {n-1 \choose 1} + \cdots + {n-1 \choose n-1}=2^{n-1}.
$$
This completes the proof.
\end{proof}

\subsection[PIE]{Principle of inclusion and exclusion}

In the following section we are going to introduce the "Principle of Inclusion and Exclusion (PIE)". Let us introduce this topic
with an example.

\begin{exmp}
Suppose that we have a room full of students. $25$ of them speak spanish, $24$ of them speak french and $8$ of them speak both spanish 
and french. How many students speak at least one foreign language? Denote the set of students speaking spanish by $S$ and the 
set of students speaking french by $F$. Now, consider the following Venn-diagram.
\begin{figure}[H]
    \centering
    \scalebox{0.6}{\begin{tikzpicture}
        \begin{scope}[shift={(3cm,-5cm)}, fill opacity=0.45]
            \fill[green] \secondcircle;
            \fill[blue] \thirdcircle;
            \draw \secondcircle node [above] {$F$};
            \draw \thirdcircle node [below] {$S$};
        \end{scope}
    \end{tikzpicture}}
    \caption{Venn-diagram for two languages.}
\end{figure}

The number of students that speak at least one language is equal to $|S \cup F|=|S| + |F|-|S \cap F|=25+24-8=41$. Notice that we must substract their intersection 
or we count the students speaking both french and spanish twice.\\\\
Now, suppose that $15$ students speak german, $6$ speak german and spanish, $7$ speak german and french and 
$4$ speak all three languages. Let us denote the set of students who speak german by $G$, the set of students who speak french by $F$ 
and the set of students that speak spanish by $S$. Consider the following Venn-diagram.

\begin{figure}[H]
    \centering
    \scalebox{0.6}{\begin{tikzpicture}
        \begin{scope}[shift={(3cm,-5cm)}, fill opacity=0.45]
            \fill[red] \firstcircle;
            \fill[green] \secondcircle;
            \fill[blue] \thirdcircle;
            \draw \firstcircle node[below] {$G$};
            \draw \secondcircle node [above] {$F$};
            \draw \thirdcircle node [below] {$S$};
        \end{scope}
    \end{tikzpicture}}
    \caption{Venn-diagram for three languages.}
\end{figure}
Now the result is very similar to the one before. The number of students that speak at least one language is equal to the following.
$$
|G \cup S \cup F|=|G| + |F| + |S| - |S \cap F| - |S \cap G| - |G \cap F| + |S \cap F \cap G|=25+24+15-8-7-6+4=47.
$$
\end{exmp}

Let us generalize this by the following theorem.

\begin{theorem}(Principle of inclusion-exclusion)
Let $A_1\cdots, A_n$ be finite subsets of a set $X$. Then 
\begin{align*}
|A_1 \cup \cdots \cup A_n|&=|A_1|+\cdots+|A_n|\\
&=\sum_{1 \leq i < j \leq n} |A_i \cap A_j|+\sum_{1 \leq i < j \leq n} |A_i \cap A_j \cap A_k| - \cdots + (-1)^{n-1}|A_1 \cap \cdots \cap A_n|.
\end{align*}
\end{theorem}

\begin{proof}
Let $x \in A_1 \cup \cdots \cup A_n$. How many times does $x$ appear on the right-hand-side? Suppose that $x \in A_1 \cap \cdots \cap A_p$ and 
$x \notin A_{p+1}, A_{p+2}, \cdots, A_n$. Then $x$ contributes to the right-hand-side with multiplicity 
$$
p- {p \choose 2} + {p \choose 3} - {p \choose 4} + \cdots + (-1)^{p-1} {p \choose p}=1+(-1 +p- {p \choose 2} + {p \choose 3} - \cdots + (-1)^{p-1} {p \choose p})=1.
$$
\end{proof}

The standard proof of the above is via induction. This is left as an exercise for the reader.

\subsection{The derangement problem}

In this section we want to apply the principle of inclusion and exclusion to the so called derangement problem. But first, we note that sometimes it turns out that the 
problem is symmetric in the following sense that $|A_{i1} \cap \cdots \cap A_{ik}|$ does not depend on $A_{i1}, \cdots, A_{ik}$, but only on $k$, that is $A^k=|A_{i1} \cap \cdots \cap A_{ik}|$. 
Then, 
$$
|A_1 \cup \cdots \cup A_n|=n \cdot A - {n \choose 2} A^2 + {n \choose 3} A^3 - \cdots + (-1)^{n-1} A^n.
$$

Now, suppose that $n$ people leave their coats in the coatroom, each of them get back a random coat. 
What is the probability that everyone will get someone else's coat? Recall that the total number of permutations 
(that is the total number of configurations of coats) is given by $n!$. We might ask: How many permutations have 
not fixed points, i.e. never map a number into itself?\\
\\
We compute the number of permutations with fixed points. Define 
$$
A_i=\{\text{permutations  bringing }i \text{ to } i\}
$$ 
and 
$$
A_{i1} \cap \cdots \cap A_{ik}=\{\text{permutations bringing } i_1 \mapsto i_i, \cdots, i_k \mapsto i_k\}.
$$
Then, $|A_i|=(n-1)!$ and $|A_{i1} \cap \cdots \cap A_{ik}|=A^k=(n-k)!$, hence
\begin{align*}
|A_1 \cup \cdots \cup A_n|&=n(n-1)!- {n \choose 2} (n-2)! + \cdots + (-1)^{n-1} \cdot 0!\\
&=
n! - \frac{n!}{2!}+ \frac{n!}{3!} - \cdots + \frac{(-1)^{n-1}n!}{n!}
\end{align*}
It follows that
$$
n! - |A_1 \cup \cdots \cup A_k|=n!\biggr(1-1 + \frac{1}{2!}- \frac{1}{3!}+ \cdots + \frac{(-1)^n}{n!}\biggl)
$$

Thus, the desired probability follows from the following equations.
\begin{align*}
p(n)&=1-1+\frac{1}{2!}-\frac{1}{3!}+\cdots + \frac{(-1)^n}{n!}\\
e^x&=1+x+\frac{x^2}{2!}+\frac{x^3}{3!}+ \cdots + \frac{x^n}{n!} + \cdots\\
&\implies \lim_{n \to \infty} p(n)=1-1+\frac{1}{2!}-\frac{1}{3!}+ \cdots = e^{-1}=\frac{1}{e}
\end{align*}
where $e=2.71828 \cdots$, thus $\frac{1}{e} \approx 0.37$.
\subsection{Exercises}

\begin{enumerate}
    \item Find the number of lattice paths from $(-2, -2)$ to $(3,3)$ passing through the point $(0,0)$ (each segment of a path goes north or east).
    \item In how many ways is it possible to permute the letters in the word \textit{EUCLID} in such a way that the order of the vowels (\textit{E,U,I}) is unchanged?
    \item Find the number of merry-go-rounds formed by $10$ carriages of two different colours.
    \item Find the number of positive integers not exceeding $300$ and \textbf{not} divisible by any of $2,3,$ and $5$.
\end{enumerate}