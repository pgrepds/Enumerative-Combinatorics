\section[Permutations, binomial coefficients]{Permutations and binomial coefficients continued}\label{permutationsBinomialCoefficient} 

The following lecture discusses some more problems in which binomial coefficients arise and will lead us 
to the famous principle of exclusion and inclusion.

\subsection{The cab driver problem}

We will begin with the so called 'cab driver problem'. Suppose that we have a city with streets arranged in 
a regular grid (like Manhattan) and suppose that our city contains $n$ blocks from north to south and $m$ blocks
from west to east. Now, suppose that we have a cab driver that wants to drive from one point on the grid to another
point, say from $A$ to $B$. In figure \ref{fig:cabdriverprobleminitialexample} the general setup is visualized.

\begin{figure}[ht]
    \centering
    \scalebox{.6}{\incfig{cabdriverprobleminitialexample}}
    \caption{The Cap Driver Problem.}
    \label{fig:cabdriverprobleminitialexample}
\end{figure}
\noindent
\textbf{Question:} In how many ways can we get from $A$ to $B$? We are only allowed to go east or north and we are not allowed to go backwards.
\\
\\
Let $P_{m,n}$ be the number of paths from $A$ to $B$. We notice first, that we can solve this problem by considering a sequence of letters. Let 
us denote "going north" by the letter "N" and "going east" by the letter "E", then a path from $A$ to $B$ corresponds to a sequence of N's and E's
of length $m+n$ with $n$ N's and $m$ E's. The total number of such words is equal to the total number of paths from $A$ to $B$.
\\
\\
There is also another solution. Starting from any point in the above grid, there are only two choices: go north or go east. Let $A$ be the starting point
and denote "going north in the first step" by $A'$ and "going east in the second step" by $A''$, then the total number of paths from $A$ to $B$ is equal to 
the total number of paths from $A'$ to $B$ plus the total number of paths from $A''$ to $B$. The result is a so called recurrence relation given by the following equation.
$$
P_{A \to B}=P_{A' \to B} + P_{A'' \to B}
$$
In general, if we go one step north the first time, then there are $P_{m, n-1}$ paths left from this position to $B$. Similarly, if we go east first, then there are $P_{m-1, n}$
paths left from this position to $B$. This results in the following relation.
$$
P_{m, n}=P_{m,n-1}+P_{m-1, n}
$$
Given the initial conditions $P_{0,n}=P_{m,0}=1$, and $P_{m,1}=m+1$ and $P_{1,n}=n+1$, our final solution can be found quickly.
$$
P_{m,n}={m + n \choose m}={m+n \choose n}
$$

\subsection{Balls in boxes and multisets}

Suppose that we have $k$ balls and $n$ boxes. In how many ways can we put $k$ balls into $n$ boxes? If the maximal capacity of each box is $1$, the answer would be 
$n \choose k$ if $n \geq k$ and $0$ otherwise. What happens if there are no restrictions on the capacity of boxes?
\\
This question leads us to the definition of a multiset. Let $X$ be a set.

\begin{defn}
A multiset is a function $\mu: X \to \mathbb{Z}_{\geq  0}$. The size of a multiset is $\sum_{x \in X} \mu (x)$.
\end{defn}q

\begin{exmp}
Assume that we have $5$ boxes, that is $X=\{1, 2, 3, 4 ,5\}$, and $\mu(1)=1, \mu(2)=0, \mu(3)=3$ and $\mu(4)=1$. Compare this with figure \ref{fig:multisetboxesexample}. Notice that box $3$ contains 
$3$ balls. The size of $\mu$ is $5$.

\begin{figure}[ht]
    \centering
    \scalebox{0.4}{\incfig{multisetboxesexamples3}}
    \caption{Balls in boxes and mutlisets.}
    \label{fig:multisetboxesexample}
\end{figure}
\end{exmp}

\subsection{Integer compositions}

\subsection{Principle of inclusion and exclusion}

\subsection{The derangement problem}

\subsection{Exercises}

\begin{enumerate}
    \item Find the number of lattice paths from $(-2, -2)$ to $(3,3)$ passing through the point $(0,0)$ (each segment of a path goes north or east).
    \item In how many ways is it possible to permute the letters in the word \textit{EUCLID} in such a way that the order of the vowels (\textit{E,U,I}) is unchanged?
    \item Find the number of merry-go-rounds formed by $10$ carriages of two different colours.
    \item Find the number of positive integers not exceeding $300$ and \textbf{not} divisible by any of $2,3,$ and $5$.
\end{enumerate}